\documentclass[12pt]{report}
\usepackage[utf8]{inputenc}
\usepackage{graphicx}
\usepackage{geometry}
  \geometry{
    left = 20mm,
    right = 20mm
  }

\usepackage{tikz}
\usetikzlibrary{arrows, shapes, positioning}

\graphicspath{{images/}}

\usepackage{amsthm}
\usepackage{amsmath}
\usepackage{amsfonts}
\usepackage{algorithmic}
\usepackage{algorithm}

\newtheorem{thm}[algorithm]{Theorem}
\newtheorem{definition}[algorithm]{Definition}
\newtheorem{cor}[algorithm]{Corollary}
\newtheorem{conj}[algorithm]{Conjecture}
\newtheorem{remark}[algorithm]{Remark}
\newtheorem{claim}[algorithm]{Claim}
\newtheorem{lemma}[algorithm]{Lemma}
\newtheorem{guess}[algorithm]{Statement}
\newtheorem*{conj*}{Conjecture}

\begin{document}
\begin{titlepage}
  \begin{center}
    \textsc{
      \large
      Eötvös Loránd University \\
      Faculty of Science \\
    }
    \vspace{0.4cm}

    \rule{14cm}{0.4pt}
    \vspace*{1.2cm}

    \LARGE{Hanna Gábor}

    \huge
    \vspace{0.6cm}
    \textbf{About a conjecture of Goddard and Henning} \\
    \vspace{0.3cm}
    \normalsize
    Diploma Thesis \\
    \vspace{1.5cm}

    \large
    Supervisor: \\
    \vspace{0.3cm}
    Kristóf Bérczi \\
    Department of Operations Research

    \vfill
    % TODO: Do I need to put here something about EFOP?
    \includegraphics[width=0.4\textwidth]{coat_of_arms.jpg} \\
    \vspace{0.1cm}
    \normalsize
    Budapest, 2018
  \end{center}
\end{titlepage}


\newpage\null\thispagestyle{empty}\newpage

\tableofcontents

\chapter*{Introduction}
Goddard and Henning formulated a conjecture about triangulated planar graphs in \cite{gh}.
In this thesis, we are going to investigate that conjecture. It states the following.

\begin{conj*}
  If $G$ is a simple triangulated planar graph of order at least $4$, then $G$ has
  a $2$-coloring such that every vertex of $G$ has neighbors in both color classes.
\end{conj*}

We refer to these kind of colorings as \emph{$2$-coupon colorings}. The conjecture can
be phrased with the notion of total dominating sets as well. We call a subset $S$ of
the vertices a \emph{total dominating set} in the graph if every vertex of $G$ has neighbors both in
$S$ and in its complement. The \emph{total dominating number} of $G$ is the maximum number
of disjoint total dominating sets in $G$.

In Chapter \ref{ch:arbitrary}, we will show that for graphs in general, it is NP-complete
to decide whether its total domatic number is at least $2$. We will also take a look
at the connection between the total domatic number and the minimum degree in the graph.

In Chapter \ref{ch:reformulating} and \ref{ch:spec}, we narrow our investigation to
the case of planar triangulations. In Chapter \ref{ch:reformulating}, we formulate
statements that are equivalent with the Goddard-Henning conjecture, or are slightly stronger.
The conjecture is proven for several special cases. In Chapter \ref{ch:reformulating}, we show
some of these, which have relatively easy proofs, whereas in Chapter \ref{ch:spec}, we go deeper
and show more complex proofs. We describe the proof of Nagy \cite{outerplanar} for Hamiltonian graphs,
and prove another special case connected to this proof. Namely, the case when the graph admits
a $2$-factor that does not contain cycles of length $4k +2$. We also give a proof of
the conjecture in the case when the graph does not have too many low-degree vertices.
The proof is based on a theorem of Zvorák and Král \cite{hypergraph} about colorings of hypergraphs.
As Barnette's \cite{barnette} following conjecture is also connected to the Goddard-Henning conjecture,
we give a brief overview of it. Barnette's conjecture states the following.

\begin{conj*}
  Every $3$-connected $3$-regular bipartite planar graph is Hamiltonian.
\end{conj*}

Finally, in Chapter \ref{ch:alg}, we show the algorithm of Avis that generates all
triangulations without too many repetitions. With the help of this algorithm, one
can verify the Goddard-Henning conjecture for small graphs.


\chapter{Coupon coloring in arbitrary graphs} \label{ch:arbitrary}
In this chapter we will examine the so-called total domatic number of graphs.

Let $G = (V, E)$ be a graph without isolated vertices.

\begin{definition}
  $S \subseteq V$ is a total dominating set if every vertex has a neighbor in
  $S$. The total domatic number of $G$ is the maximum number of disjoint total
  dominating sets.
\end{definition}

Sometimes it is more convenient to look at total dominating sets as color classes.

\begin{definition}
  A coloring of the vertices is called a $k$-coupon coloring if every vertex
  has a neighbor from each color class. The coupon coloring number of $G$ is
  the maximum $k$ for which a $k$-coupon coloring exists. The coupon coloring
  number is denoted by $\chi_c(G)$.
\end{definition}

\section{Complexity}

It turns out that determining the total domatic number (or equivalently the
coupon coloring number) of a graph is rather hard. Even determining if the total
domatic number of a graph is at least $2$ is NP-complete. We prove this by showing
that a variant of 3SAT is reducible to this question in polynomial time.

\begin{definition}
  An instance of the not-all-equal 3-satisfiability (NAE-3SAT) problem consists of
  a set $C$ of clauses on a finite set $X$ of Boolean variables, where each clause
  contains three literals. The question is whether there is a truth assignment for
  $X$ that satisfies all the clauses in $C$ such that each clause contains a false
  literal.
\end{definition}

\begin{thm}
  NAE-3SAT is NP-complete.
\end{thm}
\begin{proof}
  It can be checked in polynomial time whether a given truth assignment meets the
  requirement, so NAE-3SAT is in NP.

  To prove NP completeness, we show first a reduction from 3SAT to NAE-4SAT.
  Let $C$ be the set of clauses and $X$ be the set of variables in an instance of
  3SAT. Let $X' = X \cup y$, and $C' = \{(x_1 \vee x_2 \vee x_3 \vee y)\ |\
  (x_1 \vee x_2 \vee x_3) \in C\}$. We claim that the NAE-4SAT problem defined
  by $(C', X')$ is satisfiable if and only if the 3SAT problem defined by $(C, X)$
  is satisfiable. If the 3SAT formula is satisfied by a truth assignment then
  the same assignment with assigning the value false to $y$ satisfies the NAE-4SAT
  problem. Now suppose a truth assignment satisfies the NAE-4SAT problem. If $y$
  has value false, then the same assignment of $X$ satisfies the 3SAT problem.
  If $y$ has value true, then changing every truth value in $X$ to its opposite
  gives a truth assignment satisfying the 3SAT formula.

  Finally, the reduction from NAE-4SAT to NAE-3SAT is by adding clauses
  $(x_1 \vee x_2 \vee z)$ and $(\bar{z} \vee x_3 \vee y)$ instead of each clause
  $(x_1 \vee x_2 \vee x_3 \vee y) \in C'$.
\end{proof}

\begin{thm}
  It is NP-complete to decide whether the total domatic number of
  a graph is at least $2$.
\end{thm}

\begin{proof}
  Given a partition of the vertices into $2$ sets, it can be checked in polynomial
  time whether these sets are total dominating sets. So the problem is a member
  of NP.

  For proving NP-completeness, we will show that NAE-3SAT is reducible to this problem
  in polynomial time. Let $C$ be the set of clauses and $X$ be the set of variables
  in an instance of NAE-3SAT. We can assume that every variable $x$ appears in at least
  one clause. Otherwise we add a new clause containing $x$ and $\bar{x}$ to
  the formula. Now we construct the corresponding graph $G$. For each variable $x$,
  introduce $3$ vertices $x_1, x_2, x_3$, and $2$ edges $x_1x_2, x_2x_3$. For each
  clause $c$, introduce a vertex $c$. If $x$ is a literal in $c$, then add the edge
  $cx_1$ to the graph. If $\bar{x}$ is a literal in $c$, then add the edge $cx_3$.

  Suppose $G$ has a partition into $2$ disjoint total dominating sets: $T$ and $F$.
  Assign the value true for each variable $x$ with $x_1 \in T$ and assign the value
  false otherwise. For any variable $x$, $x_1$ and $x_3$ are the only neighbors
  of $x_2$, so $x_1$ and $x_3$ must be in different sets of the partition. If $c$
  is a vertex corresponding to a clause, then it must have neighbors both in $T$
  and $F$, and so the literals in $c$ cannot be all true nor false.

  Suppose now that the variables have a truth assignment such that each clause
  contains both true and false literals. Define $T$ and $F$ as follows. Put all
  the vertices corresponding to clauses into $T$. For each variable $x$ put $x_2$
  into $F$. Furthermore, if true was assigned to $x$, then put $x_1$ into $T$, $x_3$
  into $F$, and conversely otherwise.
\end{proof}

Let us note that the constructed graph in the proof is always a bipartite
graph.

\begin{cor}
  It is NP-complete to decide whether the total domatic number of
  a bipartite graph is at least $2$.
\end{cor}

\section{Degree restrictions}

A natural question is whether graphs with an appropriately big minimum degree
always have a total domatic number of at least $2$.

\begin{thm}
  For every $d$ there exists a graph with minimum degree $d$ and without $2$
  disjoint total dominating sets.
\end{thm}
\begin{proof}
  TODO
\end{proof}

TODO: Other degree stuff (k-regular, maxdeg-mindeg small enough, etc)


\chapter{Introducing the Goddard-Henning conjecture} \label{ch:reformulating}
As a reminder: the Goddard-Henning conjecture states that every simple triangulated
planar graph of order at least $4$ has total domatic number at least $2$. In this
chapter we try to find (TODO: find a better phrase) equivalent statements to the
conjecture, as well as (hopefully) slightly stronger statements. Even if the stronger
statements are not true, they can be useful for proving the conjecture in special cases.
(e.g. Theorem \ref{thm:odd})

\begin{definition}
  Let $G$ be a triangulated planar graph. For a vertex $v$, each triangle
  containing $v$ has an edge not containing $v$. We call the set of these edges a wheel.
\end{definition}

\begin{figure}[ht]
  \centering
  \includegraphics[width=70mm]{wheel}
  \caption{The green edges form the wheel defined by $A$ }
  \label{fig:wheel}
\end{figure}

\begin{guess}\label{s:bipartite}
  Let $G = (V, E)$ be a simple triangulated graph of order at least $4$. Then there
  exists a bipartite $G' = (V, E')$ subgraph of $G$, such that $E'$ contains at
  least one edge from each wheel of $G$.
\end{guess}

\begin{claim}
  The Goddard-Henning conjecture holds if and only if Statement \ref{s:bipartite} holds.
\end{claim}
\begin{proof}
  Let $G$ be a triangulated graph. Suppose first that it has a $2$-coupon coloring.
  $$E' = \{uv \in E\ |\ u\ \textrm{and}\ v\ \textrm{are in different color classes}\}$$
  defines a bipartite subgraph of $G$ that contains at least one edge from each wheel.

  Now suppose that there exists bipartite subgraph that meets our requirement. Color
  the vertices in one of the classes to black, and the vertices in the other class
  to white. This is a $2$-coupon coloring of the original graph.
\end{proof}

\begin{guess}\label{s:forest}
  Let $G = (V, E)$ be a simple triangulated graph of order at least $4$. Then there
  exists a forest in $G$ containing at least one edge from each wheel.
\end{guess}

\begin{remark}
  If Statement \ref{s:forest} holds, then Statement \ref{s:bipartite} also holds.
\end{remark}

\begin{guess}\label{s:no_iso}
  Let $G = (V, E)$ be a simple triangulated graph of order at least $4$. Then there
  exists a subgraph of $G'' = (V, E'')$ having the following two properties.
  \begin{enumerate}
    \item $E""$ contains exactly $1$ edge from each face of $G$.
    \item There are no isolated vertices in $G''$.
  \end{enumerate}
\end{guess}

\begin{lemma}
  A connected planar graph is bipartite if end only if each of its faces have an
  even number of edges.
\end{lemma}
\begin{proof}
  Suppose that the graph is not bipartite and thus there exists a circle $C$ of odd
  length. We show that then exists an odd face. The proof goes by induction on the
  number of faces in the inner side of $C$. If $C$ is a face, then we are done.
  If $C$ is not a face, then there exists a face $F$ in the inner side of $C$ having at least one
  common edge with $C$. $F$ does not contain every edge of $C$, since $G$ is connected.
  Let $C'$ be the symmetric difference of the edge sets of $C$ and $F$. By the parity of $C$
  either $F$ is an odd face or $C'$ is an odd circle containing less faces in its
  inner side than $C$.

  The other direction is trivial.
\end{proof}

\begin{claim}
  If Statement \ref{s:no_iso} holds, then Statement \ref{s:bipartite} (and thus
  the Goddard-Henning conjecture) also holds.
\end{claim}
\begin{proof}
  We show that $G' = (V, E - E'')$ is a subgraph required by Statement \ref{s:bipartite}.
  $G'$ is a bipartite graph, as each of its faces have $4$ edges.
  TBD
\end{proof}

TBD
TODO: Add also a figure about these statements.


\chapter{Proofs for special cases of the Goddard-Henning conjecture} \label{ch:spec}
\section{Outerplanar and Hamiltonian graphs}
Zoltán Lóránt Nagy \cite{outerplanar} showed that the conjecture of Goddard and Henning holds for
Hamiltonian graphs. For this, he characterized the $2$-coupon colorable
maximal outerplanar graphs first.

\begin{definition}
   A graph is outerplanar if it has a planar drawing for which all vertices
   belong to the outer face. A maximal outerplanar graph is an outerplanar graph
   such that adding any edge results in a non-outerplanar graphs.
\end{definition}
\begin{remark}
  The outer face of a maximal outerplanar graph is a Hamiltonian cycle.
\end{remark}

In order to provide the mentioned characterization we need to introduce a few
notions first.
\begin{definition}
  Let $G$ be a maximal outerplanar graph of order $n \ge 3$. The $M(G)$ sun graph
  of $G$ is obtained by gluing a triangle to each edge of the outer face.
\end{definition}
\begin{remark}
  $M(G)$ is a maximal outerplanar graph with $2n$ vertices, from which $n$ has
  degree $2$.
\end{remark}
\begin{remark}
  If $G$ has an odd number of vertices, then $M(G)$ does not have two disjoint
  total dominating sets, as in a $2$-coupon coloring of $M(G)$ the vertices of
  $G$ must have alternating colors. The graph on figure $\ref{fig:sungraph}$ is
  the sun graph of the $BDE$ triangle.
\end{remark}
\begin{definition}
  A vertex $v$ of a maximal outerplanar graph is called a central vertex if the
  following $3$ conditions hold.
  \begin{enumerate}
    \item $deg(v) \ge 3$
    \item Every neighbor of $v$ has degree at least $3$.
    \item For every $u, w$ neighbors of $v$ the length of the $uw$ path on the
    outer face not containing $v$ is divisible by $4$.
  \end{enumerate}
\end{definition}
\begin{claim}
  The outer face of a maximal outerplanar graph does not contain two consecutive
  central vertices.
\end{claim}
\begin{proof}
  Suppose there exists an $uv$ edge on the outer face such that $u$ and $v$ are
  central vertices. Because of the maximality of the graph there exists a $uvw$
  triangle. Index the vertices along the outer face form $v = v_1$ to $u = v_{n}$.
  Suppose $w = v_i$. From the centrality of $v$ follows that $i \equiv 2\
  (\textrm{mod}\ 4)$. On the other hand, $u$ is also a central vertex, hence
  $i \equiv 1\ (\textrm{mod}\ 4)$.
\end{proof}
\begin{definition}
  A generalized sun graph is a maximal outerplanar graph of order
  $n \equiv 2\ (\textrm{mod}\ 4)$ such that the number of
  degree $2$ vertices plus the number of central vertices is $n/2$.
\end{definition}
\begin{remark}
  Every second vertex of the outer face in a generalized sun graph is either
  central or has degree $2$.
\end{remark}

The key characterization theorem is the following.

\begin{thm}\label{thm:outerplanar}
  Let $G$ be a maximal outerplanar graph. $G$ admits $2$ disjoint total dominating
  sets if and only if $G$ is not a generalized sun graph.
\end{thm}

For proving this theorem we need some observations about generalized sun graphs.

\begin{definition}
  Let $G$ be a maximal outerplanar graph. We say that a $uv$ edge is a chord
  of length $i$, if there is a $uv$ path of length $i$ on the outer face. (This means
  that if $uv$ is a chord of length $i$, then it is also a chord of length $n - i$.)
\end{definition}

\begin{lemma} \label{lem:two_chord}
  A maximal outerplanar graph of order $n \ge 3$ has a chord of length $2$.
\end{lemma}
\begin{proof}
  It is trivial for $n = 3$. Suppose $n \ge 4$ and
  let $uv$ be a chord of minimal length among chords of length at least $2$. By
  the maximality of the graph there exists a $uvw$ face, where $w$ is on the shorter
  $uv$ path of the outer face. If $uv$ is not a chord of length $2$, then $uw$ or $vw$
  is a chord of length at least $2$ and less than the length of the $uv$ chord.
\end{proof}

\begin{lemma} \label{lem:3-4_chord}
  A maximal outerplanar graph of order $n \ge 5$ has a chord of length $3$ or $4$.
\end{lemma}
\begin{proof}
  Let $uv$ be a chord of minimal length among chords of length at least $3$. By
  the maximality of the graph there exists a $uvw$ face, where $w$ is on the $uv$ path
  on the outer face that defines the length of the chord. If on this path $w$ has a distance
  bigger than $2$ from either $u$ or $v$, than $uw$ or $vw$ is a chord contradicting
  the minimality of $uv$. Thus, the length of the $uv$ path is at most $4$.
\end{proof}

\begin{lemma} \label{lem:del_face}
  If $G$ is a maximal planar graph of order $n \ge 7$, then there exists a bounded
  face, such that the deletion of this face divides $G$ into three graphs with the
  following properties.
  \begin{enumerate}
    \item At most one of the three graphs has more than $3$ bounded faces.
    \item At least one of the three graphs has $2$ or $3$ faces.
  \end{enumerate}
\end{lemma}
\begin{proof}
  For $n \le 11$, the statement is easy to verify.

  For $n > 11$ delete the faces with $2$ common edges with the unbounded face. Then in the remaining
  graph $G_1$ delete the faces that now have $2$ common edges with the unbounded face.

  We claim that the remaining graph $G_2$ is not empty. $G$
  has $m = \frac{3(f - 1) + n}{2}$ edges, where $f$ denotes the number of faces.
  Then by Euler's formula $n = f + 1$, hence $G$ has at least $11$ faces. In the
  first deleting step at most $n/2$ faces are deleted, and in the second step at most
  $|G_1|/2$ faces are deleted. Thus at most $3(f + 1) / 4$ faces are deleted.
  $3(f + 1) / 4 \le f - 2$ if $f \ge 11$.

  Finally, choose a face $f_0$ from the remaining graph $G_2$, that has $2$ common edges with
  the unbounded face of $G_2$. We claim that $f_0$ has the desired properties. $f_0$ has at most
  one neighboring face in $G_2$, and one or two neighboring faces $f_1$ and maybe $f_2$ outside of $G_2$.
  Both of $f_1$ and $f_2$ has at most two neighboring faces outside of $G_1$, and
  at least one of $f_1$ and $f_2$ has at least one neighboring face outside of $G_1$.
\end{proof}

\begin{proof}[Proof of Theorem \ref{thm:outerplanar}]
  First we show that generalized sun graphs do not have $2$ disjoint total
  dominating sets. The proof goes by induction on the $n = 4k + 2$ number of vertices.

  For $k = 1$ there is only one generalized sun graph and it does not admit $2$
  disjoint total dominating sets. (Shown on figure \ref{fig:sungraph}.)

  Suppose $k \ge 2$ and $G$ is a generalized sun graph of order $4k + 2$. Index
  the vertices along the outer face from $v_1$ to $v_{4k + 2}$, such that every
  vertex with an odd index is central or has degree $2$. Let $c$ be a $2$-coloring of
  the graph. We show that $c$ cannot be a $2$-coupon coloring. The cardinality of the
  vertices implies that there must be two consecutive vertices $v_{2i}$ and $v_{2i + 2}$
  with the same color (say white). If $v_{2i + 1}$ has only white neighbors, then this
  coloring is not a $2$-coupon coloring. So suppose $v_{2i + 1}$ has a black neighbor
  $v_j$. In this case $v_{2i + 1}$ is a central vertex. The $v_{2i + 1}v_j$ edge divides the graph
  into two parts ($v_{2i + 1}v_j$ is an edge in both graphs). Both of
  these graphs are generalized sun graphs, as $v_{2i + 1}$ either remains a central
  vertex or becomes a vertex of degree $2$ in these smaller graphs, whereas other
  central vertices remain central vertices. By induction, the restriction of
  $c$ is not a $2$-coupon coloring in either of the smaller graphs. If there is
  a vertex $v_l$ with a monochromatic neighborhood in one of the smaller graphs
  and $l \neq 2i + 1,\ l \neq j$, then $v_l$ has the same neighborhood in $G$, hence
  all its neighbors are from the same color class. $v_{2i + 1}$ cannot violate the
  condition, as it was chosen in a way that it has both a black and a white neighbor
  in both graphs. Thus the only remaining case is when $v_j$ has a monochromatic
  neighborhood in both graphs. But in this case all of its neighbors are from
  the same color class as $v_{2i + 1}$, so it has a monochromatic neighborhood also
  in $G$.

  \vspace{0.4cm}

  Now we show that if a graph $G$ of order $n$ is not a generalized sun graph
  then it does have two disjoint total dominating sets.

  If $n \equiv 0\ (\textrm{mod}\ 4)$, then it is easy to find a $2$-coupon coloring:
  color the vertices along the boundary of the outer face by repeating the pattern
  $BBWW$.

  If $n \equiv 1\ (\textrm{mod}\ 4)$, then the same coloring method works, if you
  start the coloring from the right vertex. By lemma \ref{lem:two_chord} there
  exists a chord $uv$ of length $2$. Alternating colors in pairs starting from $v$
  does the job. (See Figure \ref{fig:4k+1}.)
  \begin{figure}[h]
    \centering
    \includegraphics[width=50mm]{4k+1}
    \caption{Coloring an outerplanar graph of order $4k + 1$}
    \label{fig:4k+1}
  \end{figure}

  If $n \equiv 3\ (\textrm{mod}\ 4)$, then start the coloring from a vertex next
  to $v$. (See Figure \ref{fig:4k+3}.)
  \begin{figure}[h]
    \centering
    \includegraphics[width=50mm]{4k+3}
    \caption{Coloring an outerplanar graph of order $4k + 3$}
    \label{fig:4k+3}
  \end{figure}

  Suppose $n \equiv 2\ (\textrm{mod}\ 4)$. We show by induction that if $G$ does not have $2$
  disjoint total dominating sets, then it is a generalized sun graph.\\
  The case $k = 1$ is easy to check.\\
  If $G$ has a chord $uv$ of length $3$, then $uv$ divides the graph into two parts:
  $G_1$ of order $4$ and $G_2$ of order $4k$. By alternating colors in pairs one can obtain
  $2$-coupon colorings of $G_1$ and $G_2$, where both $u$ and $v$ are colored to black
  in both graphs.\\
  If $G$ does not have a chord of length $3$, then by Lemma \ref{lem:3-4_chord}
  there exists a chord $uv$ of length $4$. Then $uv$ divides $G$ into two graphs.
  One of them must be the maximal outerplanar graph $G_5$ of order $5$. $u$ and $v$
  must be the degree $3$ vertices of $G_5$, as otherwise there would be a chord of length $3$
  in $G$. Note that in a $2$-coupon coloring $u$ and $v$ must have the same colors
  in order to create a proper neighborhood for the degree $2$ vertices of $G_5$.
  Consider the face $uvw$, where $w$ is not in $G_5$. The deletion of this face
  divides $G$ into $3$ graphs: $G'$, $G''$ and $G_5$. (It might be that $G'$ or $G''$ is
  degenerated in the sense that it consists only of one edge.) Let $G'$ be the graph containing
  $u$ and $w$. Without loss of generality we may assume that $|G''| \le |G'|$.\\
  Choose the $uv$ chord in a way, such that $|G'|$ is minimal. We may assume that
  $G'$ has at most $3$ faces by Lemma \ref{lem:del_face}, and thus $|G'| \le 5$.
  Thus, there are $4$ cases depending on the size of $G'$.

  \begin{itemize}
    \item Case $1$: $|G'| = 2$.
      In this case $G''| = 4k - 2$. If $G''$ has a $2$-coupon coloring, then
      it can easily be expanded to a $2$-coupon coloring of $G$. If $G''$ does
      not have a $2$-coupon coloring, then it is a generalized sun graph by induction.
      If $v$ is a degree $2$ or central vertex in $G''$, then color the vertices of $G''$ by
      alternating colors in pairs, starting with white from $w$, but color $v$ to
      black. Let $x$ denote the vertex before $v$. (See Figure \ref{fig:case1}.)
      \begin{figure}[h]
        \centering
        \includegraphics[width=60mm]{case1}
        \caption{Case $1$}
        \label{fig:case1}
      \end{figure}
      This way, only $v$ can have a monochromatic neighborhood in $G''$: if $v$ has degree
      $2$, then $wx$ is an edge, and if $v$ is central, then $v$ and $x$ have a common
      neighbor in $G''$ and $v$ has only white neighbors. $G_5$ can be colored to
      provide $v$ the missing color. \\

      If $v$ is neither a degree $2$ vertex nor a central vertex, then $w$ is. In
      this case $w$ is a central vertex in $G$ and thus $G$ is a generalized sun graph.
    \item Case 2: $|G'| = 3$.
      In a $2$-coupon coloring $u$ and $v$ must have the same color, whereas $u$
      and $w$ must have different colors. Let $H$ be the graph obtained from $G$
      by deleting $G_5$ and identifying $uw$ with $vw$. (See \ref{fig:case2})
      $G$ is $2$-coupon colorable if and only if $H$ is $2$-coupon colorable. On the other hand,
      $G$ is a generalized sun graph if and only if $H$ is a generalized sun graph.
      \begin{figure}[h]
        \centering
        \includegraphics[width=150mm]{case2}
        \caption{Case $2$: $G$ (left) and $H$ (right)}
        \label{fig:case2}
      \end{figure}
    \item Case $3$: $|G'| = 4$.
      If $|G'| = 4$, then $uw$ is a chord of length $3$, and that is a contradiction.
    \item Case $4$: $|G'| = 5$.
      $u$ and $w$ must be the degree $3$ vertices in $G'$, otherwise we could
      find a chord of length $3$. In a $2$-coupon coloring $u$ and $w$ must have
      the same color. Let $H$ be the graph obtained from $G$ by deleting $G_5$
      and identifying $uw$ with $vw$. (See \ref{fig:case4})
      \begin{figure}[h]
        \centering
        \includegraphics[width=150mm]{case4}
        \caption{Case $4$: $G$ (left) and $H$ (right)}
        \label{fig:case4}
      \end{figure}
      $G$ is $2$-coupon colorable if and only if $H$ is $2$-coupon colorable. On the other hand,
      $G$ is a generalized sun graph if and only if $H$ is a generalized sun graph.
  \end{itemize}
\end{proof}

\begin{remark}
  With a slight modification of the proof it can be shown that the vertices of a
  generalized sun graph cannot be colored in a way that every degree $2$ or central
  vertex has neighbors from both color classes.
\end{remark}

\begin{thm}
  Every triangulated graph with a Hamiltonian circle admits $2$ disjoint
  dominating sets.
\end{thm}
\begin{proof}
  TODO
\end{proof}

\begin{remark}
  Whitney \cite{ham1} proved that each triangulated planar graph without separating
  triangles is Hamiltonian. Helden \cite{ham5} strengthened this statement by proving that
  each triangulated planar graph with at most five separating triangles is Hamiltonian. 
\end{remark}

\section{Graphs without low-degree vertices}
TODO (Find the related article for graphs without degree 3 vertices)
TODO: hypergraphs

\section{Barnette's conjecture}

A conjecture of Barnette \cite{barnette} is the following.
\begin{conj}
  Every 3-connected cubic planar bipartite graph is Hamiltonian.
\end{conj}

By \ref{ham_dual} if the Barnette-conjecture holds, then the Goddard-Henning
conjecture also holds for Eulerian triangulations.

Alt, Payne, Schmidt and Wood \cite{spec_barnette} proved that the conjecture holds
for graphs, where the dual is an Eulerian planar triangulation and has a special $3$-coloring.

TODO


\chapter{Algorithm for generating all planar triangulations} \label{ch:alg}
It is a natural idea to check if the Goddard-Henning conjecture holds for small graphs with
computational help. The main challenge in doing so is generating all planar triangulations
on $n$ vertices.

Avis \cite{algorithm} gave an algorithm for generating all $3$-connected triangulated disks without
too many repetitions. For phrasing this more precisely, we call $(G, v_1, \dots v_r)$
an \emph{$r$-rooted triangulation} if $v_1, \dots, v_r$ are vertices of a face in $G$, and $G$
can be embedded in the plane with the following properties.

\begin{enumerate}
  \item The outer face of $G$ is formed by $v_1, \dots, v_r$, and they appear in
  this clockwise order.
  \item All the interior faces are triangles.
\end{enumerate}

The algorithm generates all $3$-connected $r$-rooted simple triangulations of order $n$ exactly once.
In this chapter, we will describe how this algorithm works.

To avoid a trivial case, suppose that $r < n$.

\begin{remark}
  Every $r$-rooted triangulation has a unique embedding to the plane in the sense that
  for every vertex $v$, the order of its neighbors is the same in all embeddings.
\end{remark}

\begin{claim} \label{c:3conn}
  An $r$-rooted triangulation is $3$-connected if and only if there exist no edges
  between two non-consecutive vertices of the outer face.
\end{claim}
\begin{proof}
  If there is an edge between two non-consecutive vertices of the outer face, then
  the endpoints of this edge clearly define a cut of the graph, so it is not $3$-connected.

  On the other hand, if $G = (V, E)$ is not $3$-connected, then there exist $u, v \in V$ such that
  removing $\{u, v\}$ from $G$, cuts the graph into $2$ parts. As $r$-rooted triangulations
  do not have cutting vertices, $u$ and $v$ must be on the outer face of $t$ and they
  are not consecutive to each other.
\end{proof}

The idea of the algorithm is as follows. First we construct an auxiliary graph $H = (T, F)$, where
$T$ is the set of $3$-connected $r$-rooted planar triangulations of order $n$. Then we
show that $H$ is connected. Finally, we perform a depth-first-search on $H$. For doing so,
we need to introduce a few notions first.

\begin{definition}
  We call an edge of a planar graph \emph{internal}, if it is not on the outer face of the graph.

  Let $t$ be an $r$-rooted triangulation and let $uv$ be an internal edge of $t$. There exist two
  faces containing $uv$: $uvx$ and $uvy$. We say that $uv$ is a \emph{transformable edge}
  if and only if $xy$ is not an edge in $t$.

  If $uv$ is transformable, we refer to the following operation as \emph{flipping} $uv$:
  delete the edge $uv$ and add the edge $xy$ to the graph. We denote the resulting
  graph with $flip(t, uv)$.

  We define the edge set of $H$ as follows. For $t_1, t_2 \in T$, $t_1t_2 \in F$ if and only
  there exists a transformable edge $e$ in $t_1$ such that $t_2 = flip(t_1, e)$.
\end{definition}

\begin{remark}
  $H$ is a well-defined undirected graph.
\end{remark}
\begin{proof}
  Suppose that $t_2$ can be obtained from $t_1$ by flipping
  $uv$ by adding the edge $xy$. Note, that $xy$ cannot be on the outer face of $t_2$.
  $uv$ is not an edge in $t_2$, so $xy$ is transformable, and flipping $xy$ results
  in the graph $t_1$.
\end{proof}

To prove the connectivity of $H$, we need the following lemma.

\begin{lemma} \label{l:trans}
  Let $v$ be a vertex in a $3$-connected $r$-rooted triangulation $G$. Suppose that $v$ is of degree
  at least $4$ and has at least one neighbor on the outer face of $G$. If $u_1, u_2, u_3, u_4$
  are consecutive neighbors of $v$ such that $u_1$ is on the outer face an $u_2$
  is not, then at least one of $vu_3$ and $u_2u_3$ is a transformable edge. Furthermore,
  the graph obtained by the transformation is also a $3$-connected $r$-rooted triangulation.
\end{lemma}
\begin{proof}
  $v_u3$ in an internal edge bounding faces $vu_2u_3$ and $vu_3u_4$. If $vu_3$
  is not transformable, then $u_2u_4$ is an edge. It is an internal edge, since $u_2$
  does not lie on the outer face. Let $xu_2u_4$ and $yu_2u_4$
  be the two faces bounded by $u_2u_4$. One of them must lie inside the triangle $vu_2u_4$,
  the other one must lie outside of it. ($v$ might be $x$ or $y$.) Thus, $u_2u_4$
  is transformable. (See Figure \ref{fig:trans}.)

  By Claim \ref{c:3conn}, to prove that the resulting graph is $3$-connected, it
  is enough to show that the new edge has an internal vertex.
  If $u_2u_4$ is the new edge, then $u_2$ is an internal
  vertex. Otherwise, the new edge is $xy$. One of them lies inside the $vu_2u_4$,
  and hence it is internal.
\end{proof}

\begin{figure}[ht]
  \centering
  \includegraphics[width=70mm]{trans}
  \caption{Lemma \ref{l:trans}}
  \label{fig:trans}
\end{figure}

\begin{thm}
  $H$ is connected.
\end{thm}
\begin{proof}
  The proof goes as follows. First, we fix an $r$-rooted triangulation $t^*$. Then
  for every $t \in T$, we construct a path starting from $t$. Finally,
  we show that each of these paths ends in $t^*$.

  Define $t^*$ as follows. Let $v_1, v_2 \dots, v_r$ form the outer face of $t^*$.
  Connect $v_{r+1}$ to all the vertices of the outer face. For $i \ge r + 2$,
  put $v_i$ on the $v_{r-1}v_rv_{i-1}$ face and connect it with all of these three vertices.
  (See figure \ref{fig:root}.)

  \begin{figure}[ht]
    \centering
    \includegraphics[width=70mm]{root}
    \caption{$t^*$ for $r=3$, $n = 6$}
    \label{fig:root}
  \end{figure}

  Let $t = (G, v_1, \dots v_r)$ be an $r$-rooted triangulation. We show now that
  Algorithm \ref{alg:path} stops in finitely many steps and constructs a path
  starting in $t$ and ending in $t^*$. Note, that when the algorithm flips an
  edge, then it is really transformable.
  \begin{algorithm}
    \caption{Construct path} \label{alg:path}
    \begin{algorithmic}
      \linespread{1.0}
      \small{
      \STATE let $p$ be an array of length $n$ /* $p[i]$ will be the $i^{th}$ vertex of the path.
      \STATE $p[0] := t$
      \WHILE{\TRUE}
        \STATE $j := j + 1$
        \STATE $i := 1$
        \WHILE{$deg(v_i) = 3$ and $i \le r - 2$}
          \STATE $i := i + 1$
        \ENDWHILE
        \IF{$i \le r - 2$}
          \STATE /* The outer face of $t$ differs from the outer face of $t^*$.
          \STATE let $v_{i - 1}, u_2, u_3, u_4$ be consecutive neighbors of $v_i$ in
          counterclockwise order
          \IF{$u_2u_4 \not\in E$}
            \STATE $p[j] := flip(t, v_iu_3)$
          \ELSE
            \STATE $p[j] := flip(t, u_2u_4)$
          \ENDIF
        \ELSE
          \STATE /* The outer face looks good, identify the possible $v_{r+1}$.
          \STATE let $w$ be the vertex connected to $v_1, \dots v_{r - 2}$.
          \STATE $a = v_1$
          \WHILE{$w$ has exactly one neighbor $b \neq a$ not on the outer face of  $t$}
            \STATE /* If $w$ was the possible $v_k$, then $b$ is the possible $v_{k + 1}$.
            \STATE $a = w$
            \STATE $w = b$
          \ENDWHILE
          \IF{$deg(w) = 3$}
            \STATE return $p$
          \ELSE
            \STATE let $v_r, u_2, u_3, u_4$ be consecutive neighbors of $w$ in
            counterclockwise order
            \IF{$u_2u_4 \not\in E$}
              \STATE $p[j] := flip(t, wu_3)$
            \ELSE
              \STATE $p[j] := flip(t, u_2u_4)$
            \ENDIF
          \ENDIF
        \ENDIF
        \STATE $t := p[j]$
      \ENDWHILE}
    \end{algorithmic}
  \end{algorithm}
  \linespread{1.3}


  We claim that after enough steps, $deg(v_1) = \dots = deg(v_{r - 2}) = 3$.
  Otherwise, $deg(v_i) \neq 3$ for some $i \le r - 2$. Let $i$ be the smallest such index.
  Then the algorithm takes $v_{i - 1}, u_2, u_3, u_4$ such that they are consecutive neighbors of $v_i$
  in counterclockwise order. If $v_iu_3$ is transformable, then the degree of $v_i$
  decreases and for every $j < i,$ $deg(v_j)$ does not change. Suppose that $v_iu_3$
  is not transformable. Then let $xu_2u_4$ and $yu_2u_4$ be the two faces bounding $u_2u_4$.
  Suppose $y = v_j$ for some $j < i$. From $deg(v_j) = 3$ it follows, that $u_4$ must be $v_{j - 1}$.
  On the other hand, by $3$-connectivity, if $u_4$ is an external vertex, then it
  is $v_{i + 1}$ and hence $i = r - 1$, which is a contradiction. So adding $xy$ to the graph
  does not change the degree of $v_j$ for any $j < i$. In the next step of the algorithm,
  $v_iu_3$ is transformable. Thus, we showed, that after enough steps, $deg(v_i) = 3$ for every $i \le r - 2$.

  If $v_1, \dots v_{r - 2}$ are all degree $3$ vertices, then there exists a vertex $w$
  such that $w$ is connected to $v_1, \dots v_r$. If $w$ has
  more than one internal neighbor, than the algorithm chooses $v_r, u_2, u_3, u_4$ to
  be consecutive neighbors of $w$ in counterclockwise order. If $wu_3$ is transformable,
  then the algorithm deletes $wu_3$ from the graph and adds $u_2u_4$ to it. By executing this step,
  the degree of $w$ decreases, and for $j \le r - 2$, $deg(v_j)$ does not change.
  The other case is when $wu_3$ is not transformable. Let $x$ and $y$ be the
  vertices of the two faces bounded by $u_2u_4$, In this case the algorithm
  deletes $u_2u_4$ from the graph and adds $xy$ to it. This does not change the
  degree of $v_j$ for $j \le r - 2$. In the next step of the algorithm $wu_3$ is
  transformable. Thus, after some steps, $w$ will have only one internal neighbor $b$.
  It follows, that $bv_r$ and $bv_{r - 1}$ are edges. With the same process, we
  achieve that $b$ has at most $1$ internal neighbor apart from $v$. It is easy to see, that
  the changes do not affect vertices outside of $bv_rv_{r - 1}$. Therefore, by iterating these
  steps we can achieve $t^*$.
\end{proof}

The proof shows that by performing a depth-first-search on $H$, one
can generate all $r$-rooted $3$-connected triangulations. To determine the neighbors
of a triangulation $t \in H$, we check for each edge $e$ in $t$ if $e$ is transformable in $t$.
For each pair $(t, e)$, this can be done in $O(n)$ time. Hence, if
$K$ denotes the number of all $r$-rooted $3$-connected triangulations on $n$ vertices, then
the algorithm finishes in $O(n^2K)$ time.


\addcontentsline{toc}{chapter}{Bibliography}
\begin{thebibliography}{9}

\bibitem{outerplanar}
Zoltán Lóránt Nagy
\textit{Coupon-coloring and total domination in Hamiltonian planar triangulations}
arXiv preprint arXiv:1708.01725

\bibitem{gh}
Wayne Goddard, Michael A. Henning
\textit{Thoroughly distributed colorings}
arXiv preprint arXiv:1609.09684

\bibitem{zelinka}
Bohdan Zelinka
\textit{Total domatic number and degree of vertices of a graph}.
Mathematica Slovaca, Vol. 39 (1989), No. 1, 7--11

\bibitem{np-complete}
Pinar Heggernes, Jan Arne Telle
\textit{Partitioning graphs into generalized dominating sets}
Nord. J. Comput. 5 (2), 128-142Nord. J. Comput. 5 (2), 128-142

\bibitem{regular}
H. Aram, S. M. Sheikholeslami, L. Volkmann
\textit{On the total domatic number of regular graphs}
Transactions on Combinatorics 1, 45-51

\end{thebibliography}


\end{document}
