\section{Outerplanar and Hamiltonian graphs}
Zoltán Lóránt Nagy \cite{outerplanar} showed that the conjecture of Goddard and Henning holds for
Hamiltonian graphs. For this, he characterized the $2$-coupon colorable
maximal outerplanar graphs first.

\begin{definition}
   A graph is outerplanar if it has a planar drawing for which all vertices
   belong to the outer face. A maximal outerplanar graph is an outerplanar graph
   such that adding any edge results in a non-outerplanar graphs.
\end{definition}
\begin{remark}
  The outer face of a maximal outerplanar graph is a Hamiltonian cycle.
\end{remark}

In order to provide the mentioned characterization we need to introduce a few
notions first.
\begin{definition}
  Let $G$ be a maximal outerplanar graph of order $n \ge 3$. The $M(G)$ sun graph
  of $G$ is obtained by gluing a triangle to each edge of the outer face.
\end{definition}
\begin{remark}
  $M(G)$ is a maximal outerplanar graph with $2n$ vertices, from which $n$ has
  degree $2$.
\end{remark}
\begin{remark}
  If $G$ has an odd number of vertices, then $M(G)$ does not have two disjoint
  total dominating sets, as in a $2$-coupon coloring of $M(G)$ the vertices of
  $G$ must have alternating colors. The graph on figure $\ref{fig:sungraph}$ is
  the sun graph of the $BDE$ triangle.
\end{remark}
\begin{definition}
  A vertex $v$ of a maximal outerplanar graph is called a central vertex if the
  following $3$ conditions hold.
  \begin{enumerate}
    \item $deg(v) \ge 3$
    \item Every neighbor of $v$ has degree at least $3$.
    \item For every $u, w$ neighbors of $v$ the length of the $uw$ path on the
    outer face not containing $v$ is divisible by $4$.
  \end{enumerate}
\end{definition}
\begin{claim}
  The outer face of a maximal outerplanar graph does not contain two consecutive
  central vertices.
\end{claim}
\begin{proof}
  Suppose there exists an $uv$ edge on the outer face such that $u$ and $v$ are
  central vertices. Because of the maximality of the graph there exists a $uvw$
  triangle. Index the vertices along the outer face form $v = v_1$ to $u = v_{n}$.
  Suppose $w = v_i$. From the centrality of $v$ follows that $i \equiv 2\
  (\textrm{mod}\ 4)$. On the other hand, $u$ is also a central vertex, hence
  $i \equiv 1\ (\textrm{mod}\ 4)$.
\end{proof}
\begin{definition}
  A generalized sun graph is a maximal outerplanar graph of order
  $n \equiv 2\ (\textrm{mod}\ 4)$ such that the number of
  degree $2$ vertices plus the number of central vertices is $n/2$.
\end{definition}
\begin{remark}
  Every second vertex of the outer face in a generalized sun graph is either
  central or has degree $2$.
\end{remark}

The key characterization theorem is the following.

\begin{thm}\label{thm:outerplanar}
  Let $G$ be a maximal outerplanar graph. $G$ admits $2$ disjoint total dominating
  sets if and only if $G$ is not a generalized sun graph.
\end{thm}

For proving this theorem we need some observations about generalized sun graphs.

\begin{definition}
  Let $G$ be a maximal outerplanar graph. We say that a $uv$ edge is a chord
  of length $i$, if there is a $uv$ path of length $i$ on the outer face. (This means
  that if $uv$ is a chord of length $i$, then it is also a chord of length $n - i$.)
\end{definition}

\begin{lemma} \label{lem:two_chord}
  A maximal outerplanar graph of order $n \ge 3$ has a chord of length $2$.
\end{lemma}
\begin{proof}
  It is trivial for $n = 3$. Suppose $n \ge 4$ and
  let $uv$ be a chord of minimal length among chords of length at least $2$. By
  the maximality of the graph there exists a $uvw$ face, where $w$ is on the shorter
  $uv$ path of the outer face. If $uv$ is not a chord of length $2$, then $uw$ or $vw$
  is a chord of length at least $2$ and less than the length of the $uv$ chord.
\end{proof}

\begin{lemma} \label{lem:3-4_chord}
  A maximal outerplanar graph of order $n \ge 5$ has a chord of length $3$ or $4$.
\end{lemma}
\begin{proof}
  Let $uv$ be a chord of minimal length among chords of length at least $3$. By
  the maximality of the graph there exists a $uvw$ face, where $w$ is on the $uv$ path
  on the outer face that defines the length of the chord. If on this path $w$ has a distance
  bigger than $2$ from either $u$ or $v$, than $uw$ or $vw$ is a chord contradicting
  the minimality of $uv$. Thus, the length of the $uv$ path is at most $4$.
\end{proof}

\begin{lemma} \label{lem:del_face}
  If $G$ is a maximal planar graph of order $n \ge 7$, then there exists a bounded
  face, such that the deletion of this face divides $G$ into three graphs with the
  following properties.
  \begin{enumerate}
    \item At most one of the three graphs has more than $3$ bounded faces.
    \item At least one of the three graphs has $2$ or $3$ faces.
  \end{enumerate}
\end{lemma}
\begin{proof}
  For $n \le 11$, the statement is easy to verify.

  For $n > 11$ delete the faces with $2$ common edges with the unbounded face. Then in the remaining
  graph $G_1$ delete the faces that now have $2$ common edges with the unbounded face.

  We claim that the remaining graph $G_2$ is not empty. $G$
  has $m = \frac{3(f - 1) + n}{2}$ edges, where $f$ denotes the number of faces.
  Then by Euler's formula $n = f + 1$, hence $G$ has at least $11$ faces. In the
  first deleting step at most $n/2$ faces are deleted, and in the second step at most
  $|G_1|/2$ faces are deleted. Thus at most $3(f + 1) / 4$ faces are deleted.
  $3(f + 1) / 4 \le f - 2$ if $f \ge 11$.

  Finally, choose a face $f_0$ from the remaining graph $G_2$, that has $2$ common edges with
  the unbounded face of $G_2$. We claim that $f_0$ has the desired properties. $f_0$ has at most
  one neighboring face in $G_2$, and one or two neighboring faces $f_1$ and maybe $f_2$ outside of $G_2$.
  Both of $f_1$ and $f_2$ has at most two neighboring faces outside of $G_1$, and
  at least one of $f_1$ and $f_2$ has at least one neighboring face outside of $G_1$.
\end{proof}

\begin{proof}[Proof of Theorem \ref{thm:outerplanar}]
  First we show that generalized sun graphs do not have $2$ disjoint total
  dominating sets. The proof goes by induction on the $n = 4k + 2$ number of vertices.

  For $k = 1$ there is only one generalized sun graph and it does not admit $2$
  disjoint total dominating sets. (Shown on figure \ref{fig:sungraph}.)

  Suppose $k \ge 2$ and $G$ is a generalized sun graph of order $4k + 2$. Index
  the vertices along the outer face from $v_1$ to $v_{4k + 2}$, such that every
  vertex with an odd index is central or has degree $2$. Let $c$ be a $2$-coloring of
  the graph. We show that $c$ cannot be a $2$-coupon coloring. The cardinality of the
  vertices implies that there must be two consecutive vertices $v_{2i}$ and $v_{2i + 2}$
  with the same color (say white). If $v_{2i + 1}$ has only white neighbors, then this
  coloring is not a $2$-coupon coloring. So suppose $v_{2i + 1}$ has a black neighbor
  $v_j$. In this case $v_{2i + 1}$ is a central vertex. The $v_{2i + 1}v_j$ edge divides the graph
  into two parts ($v_{2i + 1}v_j$ is an edge in both graphs). Both of
  these graphs are generalized sun graphs, as $v_{2i + 1}$ either remains a central
  vertex or becomes a vertex of degree $2$ in these smaller graphs, whereas other
  central vertices remain central vertices. By induction, the restriction of
  $c$ is not a $2$-coupon coloring in either of the smaller graphs. If there is
  a vertex $v_l$ with a monochromatic neighborhood in one of the smaller graphs
  and $l \neq 2i + 1,\ l \neq j$, then $v_l$ has the same neighborhood in $G$, hence
  all its neighbors are from the same color class. $v_{2i + 1}$ cannot violate the
  condition, as it was chosen in a way that it has both a black and a white neighbor
  in both graphs. Thus the only remaining case is when $v_j$ has a monochromatic
  neighborhood in both graphs. But in this case all of its neighbors are from
  the same color class as $v_{2i + 1}$, so it has a monochromatic neighborhood also
  in $G$.

  \vspace{0.4cm}

  Now we show that if a graph $G$ of order $n$ is not a generalized sun graph
  then it does have two disjoint total dominating sets.

  If $n \equiv 0\ (\textrm{mod}\ 4)$, then it is easy to find a $2$-coupon coloring:
  color the vertices along the boundary of the outer face by repeating the pattern
  $BBWW$.

  If $n \equiv 1\ (\textrm{mod}\ 4)$, then the same coloring method works, if you
  start the coloring from the right vertex. By lemma \ref{lem:two_chord} there
  exists a chord $uv$ of length $2$. Alternating colors in pairs starting from $v$
  does the job. (See Figure \ref{fig:4k+1}.)
  \begin{figure}[h]
    \centering
    \includegraphics[width=50mm]{4k+1}
    \caption{Coloring an outerplanar graph of order $4k + 1$}
    \label{fig:4k+1}
  \end{figure}

  If $n \equiv 3\ (\textrm{mod}\ 4)$, then start the coloring from a vertex next
  to $v$. (See Figure \ref{fig:4k+3}.)
  \begin{figure}[h]
    \centering
    \includegraphics[width=50mm]{4k+3}
    \caption{Coloring an outerplanar graph of order $4k + 3$}
    \label{fig:4k+3}
  \end{figure}

  Suppose $n \equiv 2\ (\textrm{mod}\ 4)$. We show by induction that if $G$ does not have $2$
  disjoint total dominating sets, then it is a generalized sun graph.\\
  The case $k = 1$ is easy to check.\\
  If $G$ has a chord $uv$ of length $3$, then $uv$ divides the graph into two parts:
  $G_1$ of order $4$ and $G_2$ of order $4k$. By alternating colors in pairs one can obtain
  $2$-coupon colorings of $G_1$ and $G_2$, where both $u$ and $v$ are colored to black
  in both graphs.\\
  If $G$ does not have a chord of length $3$, then by Lemma \ref{lem:3-4_chord}
  there exists a chord $uv$ of length $4$. Then $uv$ divides $G$ into two graphs.
  One of them must be the maximal outerplanar graph $G_5$ of order $5$. $u$ and $v$
  must be the degree $3$ vertices of $G_5$, as otherwise there would be a chord of length $3$
  in $G$. Note that in a $2$-coupon coloring $u$ and $v$ must have the same colors
  in order to create a proper neighborhood for the degree $2$ vertices of $G_5$.
  Consider the face $uvw$, where $w$ is not in $G_5$. The deletion of this face
  divides $G$ into $3$ graphs: $G'$, $G''$ and $G_5$. (It might be that $G'$ or $G''$ is
  degenerated in the sense that it consists only of one edge.) Let $G'$ be the graph containing
  $u$ and $w$. Without loss of generality we may assume that $|G''| \le |G'|$.\\
  Choose the $uv$ chord in a way, such that $|G'|$ is minimal. We may assume that
  $G'$ has at most $3$ faces by Lemma \ref{lem:del_face}, and thus $|G'| \le 5$.
  Thus, there are $4$ cases depending on the size of $G'$.

  \begin{itemize}
    \item Case $1$: $|G'| = 2$.
      In this case $G''| = 4k - 2$. If $G''$ has a $2$-coupon coloring, then
      it can easily be expanded to a $2$-coupon coloring of $G$. If $G''$ does
      not have a $2$-coupon coloring, then it is a generalized sun graph by induction.
      If $v$ is a degree $2$ or central vertex in $G''$, then color the vertices of $G''$ by
      alternating colors in pairs, starting with white from $w$, but color $v$ to
      black. Let $x$ denote the vertex before $v$. (See Figure \ref{fig:case1}.)
      \begin{figure}[h]
        \centering
        \includegraphics[width=60mm]{case1}
        \caption{Case $1$}
        \label{fig:case1}
      \end{figure}
      This way, only $v$ can have a monochromatic neighborhood in $G''$: if $v$ has degree
      $2$, then $wx$ is an edge, and if $v$ is central, then $v$ and $x$ have a common
      neighbor in $G''$ and $v$ has only white neighbors. $G_5$ can be colored to
      provide $v$ the missing color. \\

      If $v$ is neither a degree $2$ vertex nor a central vertex, then $w$ is. In
      this case $w$ is a central vertex in $G$ and thus $G$ is a generalized sun graph.
    \item Case 2: $|G'| = 3$.
      In a $2$-coupon coloring $u$ and $v$ must have the same color, whereas $u$
      and $w$ must have different colors. Let $H$ be the graph obtained from $G$
      by deleting $G_5$ and identifying $uw$ with $vw$. (See \ref{fig:case2})
      $G$ is $2$-coupon colorable if and only if $H$ is $2$-coupon colorable. On the other hand,
      $G$ is a generalized sun graph if and only if $H$ is a generalized sun graph.
      \begin{figure}[h]
        \centering
        \includegraphics[width=150mm]{case2}
        \caption{Case $2$: $G$ (left) and $H$ (right)}
        \label{fig:case2}
      \end{figure}
    \item Case $3$: $|G'| = 4$.
      If $|G'| = 4$, then $uw$ is a chord of length $3$, and that is a contradiction.
    \item Case $4$: $|G'| = 5$.
      $u$ and $w$ must be the degree $3$ vertices in $G'$, otherwise we could
      find a chord of length $3$. In a $2$-coupon coloring $u$ and $w$ must have
      the same color. Let $H$ be the graph obtained from $G$ by deleting $G_5$
      and identifying $uw$ with $vw$. (See \ref{fig:case4})
      \begin{figure}[h]
        \centering
        \includegraphics[width=150mm]{case4}
        \caption{Case $4$: $G$ (left) and $H$ (right)}
        \label{fig:case4}
      \end{figure}
      $G$ is $2$-coupon colorable if and only if $H$ is $2$-coupon colorable. On the other hand,
      $G$ is a generalized sun graph if and only if $H$ is a generalized sun graph.
  \end{itemize}
\end{proof}

\begin{remark}
  With a slight modification of the proof it can be shown that the vertices of a
  generalized sun graph cannot be colored in a way that every degree $2$ or central
  vertex has neighbors from both color classes.
\end{remark}

\begin{thm}
  Every triangulated graph with a Hamiltonian circle admits $2$ disjoint
  dominating sets.
\end{thm}
\begin{proof}
  TODO
\end{proof}

\begin{remark}
  Whitney \cite{ham1} proved that each triangulated planar graph without separating
  triangles is Hamiltonian. Helden \cite{ham5} strengthened this statement by proving that
  each triangulated planar graph with at most five separating triangles is Hamiltonian. 
\end{remark}

\section{Graphs without low-degree vertices}
TODO (Find the related article for graphs without degree 3 vertices)
TODO: hypergraphs

\section{Barnette's conjecture}

A conjecture of Barnette \cite{barnette} is the following.
\begin{conj}
  Every 3-connected cubic planar bipartite graph is Hamiltonian.
\end{conj}

By \ref{ham_dual} if the Barnette-conjecture holds, then the Goddard-Henning
conjecture also holds for Eulerian triangulations.

Alt, Payne, Schmidt and Wood \cite{spec_barnette} proved that the conjecture holds
for graphs, where the dual is an Eulerian planar triangulation and has a special $3$-coloring.

TODO
