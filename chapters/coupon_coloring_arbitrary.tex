In this chapter we will examine the so-called total domatic number of graphs.

Let $G = (V, E)$ be a graph without isolated vertices.

\begin{definition}
  $S \subseteq V$ is a total dominating set if every vertex has a neighbor in
  $S$. The total domatic number of $G$ is the maximum number of disjoint total
  dominating sets.
\end{definition}

Sometimes it is more convenient to look at total dominating sets as color classes.

\begin{definition}
  A coloring of the vertices is called a $k$-coupon coloring if every vertex
  has a neighbor from each color class. The coupon coloring number of $G$ is
  the maximum $k$ for which a $k$-coupon coloring exists. The coupon coloring
  number is denoted by $\chi_c(G)$.
\end{definition}

\begin{remark}
  We refer as proper colorings to colorings in the usual sense. I.e. colorings of
  the vertices such that for every vertex $v$, none of the neighbors of $v$ has the
  same color as $v$. 
\end{remark}

\section{Complexity}

It turns out that determining the total domatic number (or equivalently the
coupon coloring number) of a graph is rather hard. Even determining if the total
domatic number of a graph is at least $2$ is NP-complete. We prove this by showing
that a variant of 3SAT is reducible to this question in polynomial time.

\begin{definition}
  An instance of the not-all-equal 3-satisfiability (NAE-3SAT) problem consists of
  a set $C$ of clauses on a finite set $X$ of Boolean variables, where each clause
  contains three literals. The question is whether there is a truth assignment for
  $X$ that satisfies all the clauses in $C$ such that each clause contains a false
  literal.
\end{definition}

\begin{thm}
  NAE-3SAT is NP-complete.
\end{thm}
\begin{proof}
  It can be checked in polynomial time whether a given truth assignment meets the
  requirement, so NAE-3SAT is in NP.

  To prove NP completeness, we show first a reduction from 3SAT to NAE-4SAT.
  Let $C$ be the set of clauses and $X$ be the set of variables in an instance of
  3SAT. Let $X' = X \cup y$, and $C' = \{(x_1 \vee x_2 \vee x_3 \vee y)\ |\
  (x_1 \vee x_2 \vee x_3) \in C\}$. We claim that the NAE-4SAT problem defined
  by $(C', X')$ is satisfiable if and only if the 3SAT problem defined by $(C, X)$
  is satisfiable. If the 3SAT formula is satisfied by a truth assignment then
  the same assignment with assigning the value false to $y$ satisfies the NAE-4SAT
  problem. Now suppose a truth assignment satisfies the NAE-4SAT problem. If $y$
  has value false, then the same assignment of $X$ satisfies the 3SAT problem.
  If $y$ has value true, then changing every truth value in $X$ to its opposite
  gives a truth assignment satisfying the 3SAT formula.

  Finally, the reduction from NAE-4SAT to NAE-3SAT is by adding clauses
  $(x_1 \vee x_2 \vee z)$ and $(\bar{z} \vee x_3 \vee y)$ instead of each clause
  $(x_1 \vee x_2 \vee x_3 \vee y) \in C'$.
\end{proof}

\begin{thm}
  It is NP-complete to decide whether the total domatic number of
  a graph is at least $2$.
\end{thm}

\begin{proof}
  Given a partition of the vertices into $2$ sets, it can be checked in polynomial
  time whether these sets are total dominating sets. So the problem is a member
  of NP.

  For proving NP-completeness, we will show that NAE-3SAT is reducible to this problem
  in polynomial time. Let $C$ be the set of clauses and $X$ be the set of variables
  in an instance of NAE-3SAT. We can assume that every variable $x$ appears in at least
  one clause. Otherwise we add a new clause containing $x$ and $\bar{x}$ to
  the formula. Now we construct the corresponding graph $G$. For each variable $x$,
  introduce $3$ vertices $x_1, x_2, x_3$, and $2$ edges $x_1x_2, x_2x_3$. For each
  clause $c$, introduce a vertex $c$. If $x$ is a literal in $c$, then add the edge
  $cx_1$ to the graph. If $\bar{x}$ is a literal in $c$, then add the edge $cx_3$.

  Suppose $G$ has a partition into $2$ disjoint total dominating sets: $T$ and $F$.
  Assign the value true for each variable $x$ with $x_1 \in T$ and assign the value
  false otherwise. For any variable $x$, $x_1$ and $x_3$ are the only neighbors
  of $x_2$, so $x_1$ and $x_3$ must be in different sets of the partition. If $c$
  is a vertex corresponding to a clause, then it must have neighbors both in $T$
  and $F$, and so the literals in $c$ cannot be all true nor false.

  Suppose now that the variables have a truth assignment such that each clause
  contains both true and false literals. Define $T$ and $F$ as follows. Put all
  the vertices corresponding to clauses into $T$. For each variable $x$ put $x_2$
  into $F$. Furthermore, if true was assigned to $x$, then put $x_1$ into $T$, $x_3$
  into $F$, and conversely otherwise.
\end{proof}

Let us note that the constructed graph in the proof is always a bipartite
graph.

\begin{cor}
  It is NP-complete to decide whether the total domatic number of
  a bipartite graph is at least $2$.
\end{cor}

\section{Degree restrictions}

A natural question is whether graphs with an appropriately big minimum degree
always have a total domatic number of at least $2$.

\begin{thm}
  For every $d$ there exists a graph with minimum degree $d$ and without $2$
  disjoint total dominating sets.
\end{thm}
\begin{proof}
  TODO
\end{proof}

TODO: Other degree stuff (k-regular, maxdeg-mindeg small enough, etc). Or maybe not.
