As a reminder: the Goddard-Henning conjecture states that every simple triangulated
planar graph of order at least $4$ has total domatic number at least $2$. In this
chapter we try to find equivalent statements to the
conjecture, as well as (hopefully) slightly stronger statements. Even if the stronger
statements are not true, they can be useful for proving the conjecture in special cases.
(e.g. Theorem \ref{thm:odd})

\begin{definition}
  Let $G$ be a triangulated planar graph. For a vertex $v$, each triangle
  containing $v$ has an edge not containing $v$. We call the circle consisting of
  these edges a wheel.
\end{definition}

\begin{figure}[ht]
  \centering
  \includegraphics[width=70mm]{wheel}
  \caption{The green edges form the wheel defined by $A$ }
  \label{fig:wheel}
\end{figure}

\begin{guess}\label{s:bipartite}
  Let $G = (V, E)$ be a simple triangulated graph of order at least $4$. Then there
  exists a bipartite $H = (V, F)$ subgraph of $G$, such that $F$ contains at
  least one edge from each wheel of $G$.
\end{guess}

\begin{claim}
  The Goddard-Henning conjecture holds if and only if Statement \ref{s:bipartite} holds.
\end{claim}
\begin{proof}
  Let $G$ be a triangulated graph. Suppose first that it has a $2$-coupon coloring.
  $$F = \{uv \in E\ |\ u\ \textrm{and}\ v\ \textrm{are in different color classes}\}$$
  defines a bipartite subgraph of $G$ that contains at least one edge from each wheel.

  Now suppose that there exists bipartite subgraph that meets our requirement. Color
  the vertices in one of the classes to black, and the vertices in the other class
  to white. This is a $2$-coupon coloring of the original graph.
\end{proof}

\begin{guess}\label{s:forest}
  Let $G = (V, E)$ be a simple triangulated graph of order at least $4$. Then there
  exists a forest in $G$ containing at least one edge from each wheel.
\end{guess}

\begin{remark}
  If Statement \ref{s:forest} holds, then Statement \ref{s:bipartite} also holds.
\end{remark}

\begin{guess}\label{s:no_iso}
  Let $G = (V, E)$ be a simple triangulated graph of order at least $4$. Then there
  exists a subgraph of $H' = (V, F')$ having the following two properties.
  \begin{enumerate}
    \item $F'$ contains exactly $1$ edge from each face of $G$.
    \item There are no isolated vertices in $H'$.
  \end{enumerate}
\end{guess}

\begin{lemma}\label{l:bipartite}
  A connected planar graph is bipartite if end only if each of its faces have an
  even number of edges.
\end{lemma}
\begin{proof}
  Suppose that the graph is not bipartite and thus there exists a circle $C$ of odd
  length. We show that then exists an odd face. The proof goes by induction on the
  number of faces in the inner side of $C$. If $C$ is a face, then we are done.
  If $C$ is not a face, then there exists a face $f$ in the inner side of $C$ having at least one
  common edge with $C$. $f$ does not contain every edge of $C$, since $G$ is connected.
  Let $C'$ be the symmetric difference of the edge sets of $C$ and $f$. By the parity of $C$
  either $f$ is an odd face or $C'$ is an odd circle containing less faces in its
  inner side than $C$.

  The other direction is trivial.
\end{proof}

\begin{claim}
  Statement \ref{s:no_iso} holds if and only if Statement \ref{s:bipartite} (and thus
  the Goddard-Henning conjecture) also holds.
\end{claim}
\begin{proof}
  Let $H' = (V, F')$ be the subgraph required by Statement \ref{s:no_iso}.
  We show that $H = (V, E - F')$ is a subgraph required by Statement \ref{s:bipartite}.
  $H$ is a bipartite graph by Lemma \ref{l:bipartite}, as each of its faces have $4$ edges.
  Take a wheel $v_1v_2\dots v_k$ defined by a vertex $v$. $v$ is not an isolated vertex in $H'$, so
  there exists a vertex $v_i$ such that $vv_i \in F'$. As $F'$ contains exactly one edge
  from each face, $v_iv_{i + 1} \in F$.

  \vspace{0.4cm}

  Now let $H = (V, F)$ be the subgraph required by Statement \ref{s:bipartite}.
  Clearly, $F$ cannot contain all three edges of a face.
  We show that there exists an edge set $F_0$, such that $(V, F \cup F_0)$ is a bipartite
  subgraph of $G$ that contains exactly two edges of each face. Let $F_0 = \emptyset$.
  We will add edges to $F_0$ maintaining that $(V, F \cup F_0)$ is a bipartite graph.

  Suppose that there exists a face $uvw$ in $G$ with $uv, vw \notin F \cup F_0$, $wu \in F \cup F_0$.
  If $F \cup F_0 \cup \{ uv \}$ or $F \cup F_0 \cup \{ vw \}$ is bipartite, then
  add the appropriate edge to $F_0$. If adding either of these edges to $F_0$ creates
  an odd circle in $(V, F \cup F_0)$, then there exists a path $P_{uv}$ of odd
  length from $u$ to $v$ and a path $P_{vw}$ of odd length from $v$ to $w$. Thus $P{uv}
  + P{vw} + wu$ is a closed walk of odd length. But that is a contradiction as
  $(V, F \cup F_0)$ is a bipartite graph.

  Now suppose that there exists a face $uvw$ in $G$ such that none of its edges is
  contained in $F \cup F_0$. If either of its edges can be added to $F_0$ maintaining a
  bipartite graph, then put those edges in $F_0$. Otherwise there exist odd paths
  $P_{uv}, P_{vw}, P_{wu}$ as above. Concatenating these paths gives a closed walk
  of odd length and that yields a contradiction.

  $(V, F + F_0)$ clearly contains an edge from each wheel and contains two
  edges of each face of $G$. So $H' = (V, E - (F \cup F_0))$ contains exactly one edge
  from each face, and has no isolated vertices.
\end{proof}

One can phrase the Goddard-Henning conjecture in the dual graph as well.

\begin{claim}
  $G^* = (V^*, E^*)$ is the dual of a simple triangulated graph of order at least $4$
  if and only if $G^*$ is a $3$-regular $2$-edge-connected planar graph of order at least $4$.
\end{claim}
\begin{proof}
  It is trivial that $G^*$ is $3$-regular if and only if its dual is triangulated.

  It is also easy to see that a cut consisting of one edge corresponds to a loop
  edge in the dual, and a cut consisting of two edges corresponds to a pair of parallel edges.

  Finally, by $3$-regularity and using Euler's formula $$f^* = m^* - n^* + 2 =
  3n^*/2 - n + 2 = n^*/2 + 2,$$ where $f^*$, $m^*$, and $n^*$ denote the number of faces, edges
  and vertices of $G^*$. Thus the dual of $G^*$ has at least $4$ vertices if
  and only if $4 \le n^*/2 + 2$, i.e. $G^*$ has at least $4$ vertices.
\end{proof}

\begin{guess} \label{s:dual}
  Let $G^* = (V^*, E^*)$ be a $3$-regular $2$-edge-connected planar graph of order at least $4$.
  Then there exists a subgraph $H^* = (V^*, F^*)$ in $G^*$ that has the following $2$ properties.
  \begin{enumerate}
    \item It does not contain any odd cut of $G^*$.
    \item For every face $f$ of $G^*$, $H^*$ contains an edge $e$ not on $f$ that has at least one
    endpoint on $f$. We say that $e$ leaves the face $f$.
  \end{enumerate}
\end{guess}

\begin{claim}
  \ref{s:dual} is equivalent with \ref{s:bipartite}.
\end{claim}
\begin{proof}
  We show that given a subgraph $H = (V, F)$ that meets the requirements of \ref{s:bipartite},
  the edges corresponding to $F$ in the dual of $G$ form a subgraph $H^*$ required by
  \ref{s:dual}, and vice versa. It may be worth noting that the defined $H^*$ is
  not necessarily the same as the dual graph of $H$.

  It follows from the fact that circles of a planar graph correspond to minimal cutsets in the
  dual graph, that $H$ is bipartite if and only if $H^*$ does not contain any odd cut of $G^*$.

  Moreover, an edge from a wheel defined by $v$ in $G$, corresponds to an edge
  that leaves the face that corresponds to $v$ in the dual graph of $G$.
  Hence $H$ contains at least one edge from each wheel if and only if for every face
  of $G^*$, $H^*$ contains at least one edge that leaves that face.
\end{proof}

\begin{guess}\label{s:dual_comp}
  Let $G^* = (V^*, E^*)$ be a $3$-regular $2$-edge-connected planar graph of order at least $4$.
  Then there exists a subgraph that has the following $2$ properties.
  \begin{enumerate}
    \item It intersects every odd cut of $G^*$.
    \item For every face $f$ of $G^*$ it does not contain all the edges leaving $f$.
  \end{enumerate}
\end{guess}

\begin{claim}
  \ref{s:dual} holds if and only if \ref{s:dual_comp} holds.
\end{claim}
\begin{proof}
  If $H^*$ meets the requirements of either of the statements, the complementer subgraph in $G^*$
  meets the requirements of the other.
\end{proof}

A $2$-factor of a graph $G = (V, E)$ consists of disjoint circles covering $V$.
We can formulate a sufficient condition for the Goddard-Henning conjecture with
the help of $2$-factors.

\begin{guess} \label{s:2-factor}
  Let $G^* = (V^*, E^*)$ be a $3$-regular $2$-edge-connected planar graph of order at least $4$.
  Then there exists a $2$-factor not containing any of the faces.
\end{guess}
\begin{claim} \label{c:2-factor}
  If Statement \ref{s:2-factor} holds, then \ref{s:dual} holds.
\end{claim}
\begin{proof}
  Let $H^* = (V^*, F^*)$ be the $2$-factor containing none of the faces of $G^*$.

  Every cut of $G^*$ has an even number of common edges with every circle in $H^*$.
  Therefore $H^*$ does not contain any odd cuts of $G^*$.

  Let $f = v_1v_2 \dots v_l$ be a face of $G^*$. As $F^*$ does not contain $f$,
  there must exist a $v_i$ such that $v_iv_{i + 1} \notin F^*$. Moreover, every
  vertex has degree $2$ in $H^*$, so there is an edge starting from $v_i$ that leaves $f$.
\end{proof}

The existence of $2$-factors in which some cycles are not allowed, is a well-studied
part of graph theory. We will cover some of these results in Chapter \ref{ch:2-factors}.

Statement \ref{s:2-factor} can easily be converted into a statement about perfect matchings.

\begin{guess} \label{s:matching}
  Let $G^* = (V^*, E^*)$ be a $3$-regular $2$-edge-connected planar graph of order at least $4$.
  Then there exists a perfect matching containing at least one edge from each face.
\end{guess}
\begin{claim}
  Statement \ref{s:2-factor} holds if and if Statement \ref{s:matching} holds.
\end{claim}
\begin{proof}
  As $G^*$ is $3$-regular, a subgraph is a $2$-factor if and only if the complementer
  subgraph is a perfect matching. Clearly, a subgraph contains none of the faces
  if and only if the complementer subgraph does contain at least one edge from
  each face.
\end{proof}

TODO: Add a figure about these statements, mention how Barnette's conjecture fits
here.
