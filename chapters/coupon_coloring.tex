\section{...}

We will examine the so-called total domatic number of graphs.
Let $G = (V, E)$ be a graph without isolated vertices.

\begin{definition}
  $S \subseteq V$ is a total dominating set if every vertex has a neighbor in
  $S$. The total domatic number of $G$ is the maximum number of disjoint total
  dominating sets.
\end{definition}

Sometimes it is more convenient to look at total dominating sets as color classes.

\begin{definition}
  A coloring of the vertices is called a $k$-coupon coloring if every vertex
  has a neighbor from each color class. The coupon coloring number of $G$ is
  the maximum $k$ for which a $k$-coupon coloring exists. The coupon coloring
  number is denoted by $\chi_c(G)$.
\end{definition}

It turns out that determining the total domatic number (or equivalently the
coupon coloring number) of a graph is rather hard.

\begin{definition}
  NAE-3SAT ...
\end{definition}

\begin{thm}
  NAE-3SAT is NP-complete.
\end{thm}

\begin{proof}
  ...
\end{proof}

\begin{thm}
  It is NP-complete to decide whether the total domatic number of
  a graph is at least $2$.
\end{thm}

\begin{proof}
  Given a partition of the vertices into $2$ sets, it can be checked in polynomial
  time whether these sets are total dominating sets. So the problem is a member
  of NP.

  For proving NP-completeness, we will show that NAE-3SAT is reducible to this problem
  in polynomial time. Let $C$ be the set of clauses and $X$ be the set of variables
  in an instance of NAE-3SAT. We can assume that every variable $x$ appears in at least
  one clause. Otherwise we add a new clause containing $x$ and $\bar{x}$ to
  the formula. Now we construct the corresponding graph $G$. For each variable $x$,
  introduce $3$ vertices $x_1, x_2, x_3$, and $2$ edges $x_1x_2, x_2x_3$. For each
  clause $c$, introduce a vertex $c$. If $x$ is a literal in $c$, then add the edge
  $cx_1$ to the graph. If $\bar{x}$ is a literal in $c$, then add the edge $cx_3$.

  Suppose $G$ has a partition into $2$ disjoint total dominating sets: $T$ and $F$.
  Assign the value true for each variable $x$ with $x_1 \in T$ and assign the value
  false otherwise. For any variable $x$, $x_1$ and $x_3$ are the only neighbors
  of $x_2$, so $x_1$ and $x_3$ must be in different sets of the partition. If $c$
  is a vertex corresponding to a clause, then it must have neighbors both in $T$
  and $F$, and so the literals in $c$ cannot be all true nor false.

  Suppose now that the variables have a truth assignment such that each clause
  contains both true and false literals. Define $T$ and $F$ as follows. Put all
  the vertices corresponding to clauses into $T$. For each variable $x$ put $x_2$
  into $F$. Furthermore, if true was assigned to $x$, then put $x_1$ into $T$, $x_3$
  into $F$, and conversely otherwise.
\end{proof}

Let us note that the constructed graph in the proof is always a bipartite
graph.

\begin{cor}
  It is NP-complete to decide whether the total domatic number of
  a bipartite graph is at least $2$.
\end{cor}


\section{Degree restrictions}

A natural question is whether graphs with an appropriately big minimum degree
always have a total domatic number of at least $2$.

\begin{thm}
  For every $d$ there exists a graph with minimum degree $d$ and without $2$
  disjoint total dominating sets.
\end{thm}
\begin{proof}
  ...
\end{proof}

...Other degree stuff (k-regular, maxdeg-mindeg small enough)...

\section{A conjecture of Goddard and Henning}

From now on we will focus on $2$-coupon colorings and planar graphs. A
conjecture of Goddard and Henning is the following.

\begin{conj}
  If $G$ is a simple triangulated planar graph of order at least $4$, then the
  total domatic number of $G$ is at least $2$.
\end{conj}

\begin{remark}
  The simplicity of the graph is necessary. Suppose the graph on figure
  \ref{fig:parallel} has a $2$-coupon coloring. Then $A$ and $C$ must have
  different colors, because they are the only neighbors of $B$. Similarly, $C$
  and $E$ must have different colors, as well as $E$ and $A$. That is a
  contradiction, since $A$, $C$ and $E$ form a triangular.
\end{remark}

\begin{figure}[h]
  \centering
  \includegraphics[width=70mm]{parallel}
  \caption{Simplicity is necessary}
  \label{fig:parallel}
\end{figure}

\begin{remark}
  Allowing triangulated disks (i.e. planar graphs with at most one face greater
  than $3$), the conjecture does not hold. For example, the graph on figure
  \ref{fig:sungraph} does not have a $2$-coupon coloring from similar reasons as
  the previous one. We will show later that this graph is a member of a bigger
  graph family without $2$ disjoint dominating sets.
\end{remark}

\begin{figure}[ht]
  \centering
  \includegraphics[width=70mm]{sungraph}
  \caption{The conjecture does not hold for triangulated disks}
  \label{fig:sungraph}
\end{figure}

There are some sufficient conditions known for having a total domatic number of
at least $2$. We take a look at these results now.

The first example is a graph family for which an easy induction shows that they
are $2$-coupon colorable.

\begin{definition}
  A graph is called a stacked graph if it can be constructed from a triangle by
  repeatedly putting a new vertex in a face and connecting it with the vertices on
  the boundary of that face.
\end{definition}
\begin{remark}
  Stacked graphs are triangulated.
\end{remark}
\begin{claim}
  Stacked graphs with at least $4$ vertices are $2$-coupon colorable.
\end{claim}
\begin{proof}
  We can determine the colors of the vertices as the graph is constructed. The
  current coloring will maintain two following two properties.
  \begin{enumerate}
    \item It is a $2$-coupon coloring of the current graph.
    \item Every face has vertices from both color classes.
  \end{enumerate}
  The construction of the graph starts with a simple triangle. Color two vertices
  of the triangle to blue, and the remaining vertex to red. Color the vertex added
  to the graph in the first step to red. This coloring has the desired properties.
  When a vertex is inserted into a face, color the new vertex to red, if there is
  only one red vertex on the face's boundary, and blue otherwise. This trivially
  maintains the desired properties.
\end{proof}

\begin{thm}
  Let $G$ be a triangulated planar graph. If all the vertices of $G$ have an
  odd degree, then there exists a coupon coloring with $2$ colors.
\end{thm}
\begin{proof}
  ...
\end{proof}

\begin{thm}
  Outerplanar graphs ...
\end{thm}

\begin{thm}
  Every triangulated graph with a Hamiltonian circle admits 2 disjoint
  dominating sets.
\end{thm}

Hypergraph connection...
